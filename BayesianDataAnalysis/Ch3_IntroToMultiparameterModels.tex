\documentclass{article}

\usepackage[margin=0.6in]{geometry} 

\usepackage{amsmath,amsthm,amssymb,hyperref}

\usepackage{graphicx}
\usepackage[shortlabels]{enumitem}
\usepackage{tgschola}

\newcommand{\R}{\mathbf{R}}  
\newcommand{\Z}{\mathbf{Z}}
\newcommand{\N}{\mathbf{N}}
\newcommand{\Q}{\mathbf{Q}}
\newcommand{\E}{\mathbb{E}}
\newcommand{\V}{\mathbb{V}}

\newenvironment{theorem}[2][Theorem]{\begin{trivlist}
\item[\hskip \labelsep {\bfseries #1}\hskip \labelsep {\bfseries #2.}]}{\end{trivlist}}
\newenvironment{lemma}[2][Lemma]{\begin{trivlist}
\item[\hskip \labelsep {\bfseries #1}\hskip \labelsep {\bfseries #2.}]}{\end{trivlist}}
\newenvironment{claim}[2][Claim]{\begin{trivlist}
\item[\hskip \labelsep {\bfseries #1}\hskip \labelsep {\bfseries #2.}]}{\end{trivlist}}
\newenvironment{problem}[2][Problem]{\begin{trivlist}
\item[\hskip \labelsep {\bfseries #1}\hskip \labelsep {\bfseries #2.}]}{\end{trivlist}}
\newenvironment{proposition}[2][Proposition]{\begin{trivlist}
\item[\hskip \labelsep {\bfseries #1}\hskip \labelsep {\bfseries #2.}]}{\end{trivlist}}
\newenvironment{corollary}[2][Corollary]{\begin{trivlist}
\item[\hskip \labelsep {\bfseries #1}\hskip \labelsep {\bfseries #2.}]}{\end{trivlist}}

\newenvironment{solution}{\begin{proof}[Solution]}{\end{proof}}

\begin{document}

\large % please keep the text at this size for ease of reading.

% ------------------------------------------ %
%                 START HERE             %
% ------------------------------------------ %

{\Large Page 79 % Replace with appropriate page number 
\hfill  Ch3, Introduction to Multiparameter Models}

\begin{center}
{\Large Xinyu Tan} 
\end{center}
\vspace{0.05in}

% -----------------------------------------------------
% The "enumerate" environment allows for automatic problem numbering.
% To make the number for the next problem, type " \item ". 
% To make sub-problems such as (a), (b), etc., use an "enumerate" within an "enumerate."
% -----------------------------------------------------
 \renewcommand{\labelitemi}{$\textendash$}
\begin{itemize}

% -----------------------------------------------------
% Ex. 9
% -----------------------------------------------------

\item \textbf{9. Conjugate normal model}

Suppose  $y$ is an i.i.d sample of size $n$ from the distribution $N(\mu, \sigma^2)$. Prior for $(\mu, \sigma^2)$ is
$$
\text{N-Inv-} {\chi}^2 (\mu, \sigma^2 | \mu_0, \sigma_0^2/\kappa_0; \nu_0, \sigma_0^2)
$$

This is derived from 

\begin{align*}
\mu | \sigma^2 &\sim N (\mu_0, \sigma^2/\kappa_0) \\
\sigma^2 &\sim \text{Inv-} \chi^2 (\nu_0, \sigma_0^2)
\end{align*}

Notice the formula for $\text{Inv-}\chi^2 (\nu, s^2)$ is $$p(\theta) \propto s^{\nu} \theta^{-(\nu/2 + 1)} e^{1/(2\theta)}$$ and $\text{N-Inv-} {\chi}^2 (\mu, \sigma^2 | \mu_0, \sigma_0^2/\kappa_0; \nu_0, \sigma_0^2)$ $$p(\mu, \sigma^2) \propto \sigma^{-1} (\sigma^2)^{-(\nu_0/2 + 1)} \exp \left (-\frac{1}{2\sigma^2} [\nu_0 \sigma_0^2 + \kappa_0 (\mu_0 - \mu)^2] \right)$$

Therefore, the posterior 
\begin{align*}
p(\mu, \sigma^2 | y) &\propto  p(\mu, \sigma^2) \prod_{i=1}^n p(y_i | \mu, \sigma^2) \\
	&= \sigma^{-1} (\sigma^2)^{-(\nu_0/2 + 1)} \exp \left (-\frac{1}{2\sigma^2} [\nu_0 \sigma_0^2 + \kappa_0 (\mu_0 - \mu)^2] \right) \times \\
	& \qquad (\sigma^2)^{-n/2} \exp \left ( -\frac{1}{2\sigma^2}[(n-1)s^2 + n (\bar y - \mu)^2] \right )
\end{align*}

\textit{The rest arithmetic is rather dull, but I guess it's good to do it at least once.}

First, merge the terms outside the $\exp(\cdots)$:
$$
\sigma^{-1} \sigma^{-\left( (\nu_0 + n)/ 2 + 1 \right)}
$$

Let's focus on the terms inside $\exp (\cdots)$, discarding the exponential:
\begin{align*}
&\quad -\frac{1}{2\sigma^2} [\nu_0 \sigma_0^2 + \kappa_0 (\mu_0 - \mu)^2] -\frac{1}{2\sigma^2}[(n-1)s^2 + n (\bar y - \mu)^2] \\
&= -\frac{1}{2\sigma^2} \left ( \nu_0\sigma_0^2 + (n-1)s^2 + \kappa_0 (\mu_0 - \mu)^2 + n (\bar y - \mu)^2 \right) \\
&= -\frac{1}{2\sigma^2} \left ( \nu_0\sigma_0^2 + (n-1)s^2  + \kappa_0 \mu_0^2 + n {\bar y}^2 + (\kappa_0 + n) \mu^2 - 2(\kappa_0 \mu_0 + n \bar y) \mu \right ) \\
&= -\frac{1}{2\sigma^2} \left ( \nu_0\sigma_0^2 + (n-1)s^2  + \kappa_0 \mu_0^2 + n {\bar y}^2 + (\kappa_0 + n) \left( \mu - \frac{\kappa_0 \mu_0 + n \bar y}{\kappa_0 + n} \right) ^2  - \frac{(\kappa_0 \mu_0 + n \bar y)^2}{\kappa_0 + n}\right ) \\
&= -\frac{1}{2\sigma^2} \left ( \nu_0\sigma_0^2 + (n-1)s^2  + \frac{\kappa_0 n}{\kappa_0 + n} (\bar y - \mu_0)^2  +  (\kappa_0 + n) \left( \mu - \frac{\kappa_0 \mu_0 + n \bar y}{\kappa_0 + n} \right) ^2 \right)
\end{align*}

Let 
\begin{align*}
\mu_n &= \frac{\kappa_0}{\kappa_0 + n} \mu_0 + \frac{n}{\kappa_0 + n} \bar y  \\
\kappa_n &= \kappa_0 + n \\
\nu_n &= \nu_0 + n \\
\nu_n \sigma_n^2 &= \nu_0\sigma_0^2 + (n-1)s^2  + \frac{\kappa_0 n}{\kappa_0 + n} (\bar y - \mu_0)^2
\end{align*}
, then
$$
p(\mu, \sigma^2 | y) = \text{N-Inv-} \chi^2 (\mu, \sigma^2 | \mu_n, \sigma_n^2 / \kappa_n; \nu_n, \sigma_n^2)
$$



% -----------------------------------------------------
% End
% -----------------------------------------------------
\end{itemize}
\end{document}